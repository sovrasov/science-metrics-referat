\section{Введение}

Попадание в рейтинги ведущих мировых университетов является одной из приоритетных
задач для российских ВУЗов. Присутствие в таких рейтингах позволяет
проводить более успешную рекламную кампанию среди абитуриентов, а также
побеждать в конкурсных программах финансирования. В силу различных причин,
наукометрические показатели российских вузов не такие высокие, как у зарубежных.
С другой стороны, возможно, что рейтинги в существенной степени зависят не только
от формальной научной результативности университетов. В рамках данной
работы будут рассмотрены несколько мировых рейтингов ВУЗов и будет оценён
вклад значения наукометрических показателей в рассматриваемые рейтинги.

\section{Международные рейтинги высших учебных заведений}

В настоящее время наиболее известными мировыми академическими рейтингами
являются Times Higher Education (THE), QS Rankings и Academic Ranking of World Universities (Шанхайский рейтинг).
Они составляются независимыми организациями и обновляются ежегодно.

Все рассматриваемые рейтинги, на первый взгляд, похожи и имеют простую структуру:
они представляет собой линейную взывешенную сумму некоторых индикаторов, которые
измеряются выпускающей рейтинг организацией для каждого высшего учебного заведения.
Все различия рейтингов кроются в списках показателей эффективности, методиках их оценки
и вкладе, создаваемом каждым показателем.

\subsection{Рейтинг Times Higher Education}

Рейтинг выпускается британским изданием Times Higher Education и
использует 13 показателей, которые делятся на 5 групп: преподавание
(среда обучения), исследования (объем, доход и репутация), объём цитирования (влияние исследований),
международное взаимодействие (сотрудники, студенты и исследования), доход от производственной
деятельности (инноваций) (передача знаний).

\begin{enumerate}
  \item Показатель преподавание (среда обучения) включает в себя пункты:
  \begin{itemize}
    \item Академическая репутация в сфере образования (репутационный опрос) 15\%
    \item Отношение числа преподавателей к числу студентов 4.5\%
    \item Отношение числа выпускников с PhD и выпускников-бакалавров 2.25\%
    \item Отношение числа присужденных докторских степеней к числу сотрудников университета 6\%
    \item Доход университета 2.25\%
  \end{itemize}
  \item Показатель исследования (объем, доход и репутация):
  \begin{itemize}
    \item Исследовательская репутация (репутационный опрос) 18\%
    \item Доход от исследовательской деятельности 6\%
    \item Продуктивность исследований 6\%. Эта метрика подразумевает количество публикаций
    в академических журналах, индексируемых Scopus, на одного учёного, отскалированная в
    соответствии с размером университета и предметной областью.
  \end{itemize}
  \item Показатель цитирования (влияние исследований):
    средняя цитируемость работ, вычисленная по базе Scopus. При подсчёте используются
    номализованные данные, чтобы отразить различия в объеме цитирования между различными
    предметными областями. Это означает, что учреждения с высоким уровнем исследовательской
    активности по предметам с традиционно высоким показателем цитирования не получают несправедливого преимущества.
    Для работ с числом авторов, превышающем 1000, используется специальная система
    инкорпорирования в общую статистику, поскольку без специального учёта они оказывали непропорциональное
    влияние на оценки цитируемости небольшого числа университетов.
  \item Показатель международное взаимодействие (сотрудники, студенты и исследования):
  \begin{itemize}
    \item Доля иностранных студентов 2.5\%
    \item Доля иностранных сотрудников 2.5\%
    \item Международное сотрудничество (доля научных публикаций, написанных в
    соавторстве с иностранными учеными) 2.5\%. Здесь подразумевается доля
    публикаций за последние 5 лет, которые миеют хотя бы одного иностранного соавтора. Значение
    индикатора нормализовано с учётом предметных областей, в которых работает университет.
  \end{itemize}
  \item Доход от производственной деятельности (инноваций) (передача знаний) 2.5\%.
\end{enumerate}

Рассмотрев все основные показатели, можно сделать вывод, что напрямую к наукометрии относятся
продуктивность исследований, цитирование, международное сотрудничество. Суммарно
эти пункты вносят 38.5\% вклада в финальное значение рейтинга.

\subsection{Рейтинг QS Rankings}
Рассчитывается по методике британской консалтинговой компании Quacquarelli Symonds.
Рейтинги QS и THE появились в 2010ом году, заменив собой The World University Rankings,
ранее выпускавшийся совместно Quacquarelli Symonds и Times Higher Education.
Оценка лучших университетов мира в данном рейтинге производится на основе шести критериев:
\begin{enumerate}
\item Академическая репутация 40\% – опирается на мнения профессоров и преподавателей, ведущих научную деятельность, а также высшего руководства университетов, о том, в каких учебных заведениях мира научные исследования по их зоне компетенций проводятся на самом высоком уровне. Это кумулятивная оценка, учитываются данные за последние 3 года. Респонденты называют лучшие вузы в каждой области научных исследований и лучшие вузы тех регионов, с которыми они знакомы.
\item Репутация среди работодателей 10\% – приглашения к участию рассылаются по компаниям всех индустрий, размером от ста сотрудников и выше. Отвечать могут как директора по персоналу, так и топ-менеджеры, непосредственно работающие со свеженанятыми выпускниками вузов.
\item Соотношение преподавательского состава к числу студентов 20\% - источником этих данных являются не только сведения самих вузов, но и данные государственных организаций. По возможности данные проверяются по нескольким открытым источникам для большей достоверности. Учитывается число студентов полного цикла обучения и число преподавателей на полной занятости, заочники и поставочники считаются по конверсии 1 к 3. Этот показатель у российских университетов один из самых лучших в мире.
\item Индекс цитируемости 20\% - этот критерий включает в себя количество цитат из опубликованных научных исследований на число преподавателей и исследователей, работающих в вузе как в основном месте работы на протяжении как минимум одного семестра. С 2004 по 2007 цитирование высчитывалась на основе базы данных Thomson, с 2007 года на основе библиометрической базы данных Scopus. В расчет принимаются опубликованные за последние пять лет материалы, самоцитирования не учитываются.
\item Доля иностранных студентов 5\% - учитываются студенты, являющиеся гражданами стран, отличных от страны обучения, и обучающиеся на кампусе вуза на протяжении как минимум семестра и не являющиеся студентами по обмену.
\item Доля иностранных преподавателей 5\% - как и в предыдущем случае, учитываются преподаватели, работающие на условиях полной занятости либо на полставки, и проводящие в университете не менее одного семестра.
\end{enumerate}

Напрямую наукометрическим показателем из рассматриваемых рейтингом является лишь
индекс цитируемости, который вносит 20\% в финальный показатель рейтинга.

\subsection{Шанхайский рейтинг (ARWU)}

Academic Ranking of World Universities, ARWU, больше известный как Шанхайский рейтинг,
составлен азиатским агентством ShanghaiRanking Consultancy.
Первоначальной целью создания рейтинга было было желание выяснить разрыв
между китайскими университетами и университетами мирового класса, в частности,
с точки зрения академической и исследовательской деятельности.
Этот рейтинг сфокусирован на научной и академической деятельности вузов,
чтобы снизить влияние особенностей национальных систем образования на итоговую оценку.

\begin{enumerate}
  \item Выпускники университета, получившие Нобелевскую премию или медаль Филдса 10\%.
  Под выпускниками понимаются те, кто получил степень бакалавра, магистра или доктора
  в исследуемом университете. Считается только одна премия и одна степень.
  Больший удельный вес имеет число лиц, получивших степень после 1991 года,
  меньший - те, кто получил степень в период с 1901 по 1910 год.
  \item Сотрудники университета, получившие Нобелевскую премию или медаль Филдса 20\%.
  \item Высоко цитируемые исследователи в 21 широкой дисциплинарной области 20\%.
  \item Статьи, опубликованные в журналах «Nature» и «Science» 20\%.
  \item Работы, проиндексированные Расширенным индексом научного цитирования и
  Индексом цитирования в социальных науках Web of Sceince 20\%.
  \item Академическая продуктивность университета в пересчете на одного человека 10\%.
  Представляет собой соотношение пяти вышеизложенных показателей к численности персонала университета.
\end{enumerate}

Практически все учитываемые показатели рейтинга, кроме первых двух, так или иначе
связаны с цитируемостью работ или количеством публиуаций в ведущих журналах.
Доля таких показателей в общем рейтинге составляет 70\%.

\section{Сравнение объёма наукометрических показателей в разных рейтингах}
\section{Заключение}
